\documentclass[a4paper,12pt]{article}
\usepackage[utf8]{inputenc}
\usepackage[T1]{fontenc}
\usepackage[slovene]{babel}
\usepackage{graphicx}
\graphicspath{{images/}}
\usepackage{amsmath}
\usepackage[colorinlistoftodos]{todonotes}

\title{Kratka predstavitev problema iskanja najdaljše plantaže}

\author{Ema Kozin in Filip Škrlj}

\date{december 2020}

\begin{document}

\maketitle

\newpage
\section{Opis problema}

Tema najinega projekta je problem iskanja grafa na množici $n$ točk $P$ z najdaljšo dolžino, za katerega mora veljati, da je drevo in da se njegove povezave ne križajo med seboj. Temu grafu pravimo najdaljša $platana$, zaenkrat še ni znano, ali je naš problem rešljiv v polinomskem času.
Osredotočili se bomo na računanje najdaljše $dvozvezde$, tj. drevo, ki ima dve točki $a$ in $b$, taki, da v drevesu obstaja povezava $ab$ in je vsaka druga točka povezana z $a$ ali z $b$. Če se torej omejimo le na dvozvezdne grafe, postane problem rešljiv v polinomskem času z uporabo dinamičnega programiranja.



\section{Reševanje problema}
Uporabila bova naslednji algoritem za iskanje najdaljše dvozvezde s korenoma $a$ in $b$: \\
Brez škode za splošno lahko predpostavimo, da $a$ in $b$ ležita na vodoravni črti, kjer je $a$ levo od $b$. Poleg tega lahko še predpostavimo, da vse druge točke ležijo nad to črto. Da lahko to rešimo preko dinamičnega programiranja, si poglejmo naslednji podproblem, indeksiran z parom različnih točk $p$ in $q$, kjer se povezavi $ap$ in $bq$ ne sekata. Vidimo, da povezave $ap$, $pq$, $qb$ in $ba$ tvorijo štirikotnik. Z $Q(p,q)$ označimo predel tega štirikotnika pod vzporednico stranice $ab$, ki gre skozi $y = min(y(p),y(q))$.
Naj bo $Z(p,q)$ dolžina najdaljše dvozvezde s korenoma $a$ in $b$ na točkah znotraj $Q(p,q)$, brez upoštevanja dolžine $ab$. Če vsebuje vsaj štirikotnik $Q(p,q)$ kakšno točko množice $P$, poiščemo najvišje ležečo točko in jo označimo s $k_{p,q}$ ,ki jo povežemo bodisi z $a$ bodisi z $b$. Tako dobimo  trikotnika $L$ ali $R$ ($L$ dobimo, če $k_{p,q}$ povežemo z $a$, $R$ pa, če povežemo z $b$ )ter tako v nadaljevanju postavimo $k_{p,q}$ na mesto $p$-ja ali pa na mesto $q$-ja, odvisno od tega, kateri trikotnik opazujemo. Formalno dobimo naslednje: \\
\begin{equation*}
		$$\[ Z(p,q) = \left\{ \begin{array}{ll}
		0 & \mbox{če $Q(p,q)$ ne vsebuje nobene točke iz $P$};\\
		max \left\{ \begin{tabular}{ccc}
			$Z(k_{p,q},q) + \|ak_{p,q} \| + \sum_{l \in L_{p,q}}\|al\| $ \\
			$Z(p, k_{p,q}) + \|bk_{p,q}\| + \sum_{r \in R_{p,q}}\|br\| $
		\end{tabular} \right\} & \mbox{sicer}\end{array} \right. \]$$
\end{equation*}
Če zgornje uporabimo za vsak primeren par $(p, q)$ , najdemo najdaljšo dvozvezdo s korenoma $a$ in $b$.
Uporabljeni algoritem se da izvesti v času $O(n^2)$.



\section{Načrt nadaljnega dela}	
Kot programsko okolje sva si izbrala Python, ki se zaradi knjižnice Jupyter zdi primeren za generiranje in računanje večjih primerov.
Najprej bova zgenerirala točke na specifičnih odsekih ravnine po vrsti:
\begin{itemize}
	\item[a)] kvadrat
	\item[b)] kolobar
	\item[c)] zelo tanek pravokotnik
\end{itemize}
Na generiranih točkah bova nato z prej omenjenim algoritmom izračunala najdaljši dvozvezdni in zvezdni graf. Primerjala bova, kako blizu po dolžini sta si enozvezdna in dvozvezdna rešitev ter kolikšna je časovna diskrepanca med pristopoma.

Primere izračunanih grafov  ter rezultate analize bova vizualizirala s knjižnico Matplotlib.


	
	
\end{document}